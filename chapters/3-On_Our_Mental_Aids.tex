In the previous section we have stated that the programmer's duty is to
make his product ``usefully structured'' and we mentioned the program structure
in connection with a convincing demonstration of the correctness of the program.

But how do we convince? And how do we convince ourselves? What are the
typical patterns of thought enabling ourselves to understand? It is to a broad
survey of such questions that the current section is devoted. It is written with
my sincerest apologies to the professional psychologist, because it will be
amateurishly superficial. Yet I hope (and trust) that it will be sufficient to
give us a yardstick by which to measure the usefulness of a proposed structuring.

Among the mental aids available to understand a program (or a proof of its
correctness) there are three that I should like to mention explicitly:
\begin{enumerate}[1) ]
    \item Enumeration
    \item Mathematical induction
    \item Abstraction.
\end{enumerate}

\dashuline{On enumeration.}

I regard as an appeal to enumeration the effort to verify a property of the
computations that can be evoked by an enumerated set of statements performed in
sequence, including conditional clauses distinguishing between two or more cases.
Let me give a simple example of what I call ``enumerative reasoning''.

It is asked to establish that the successive execution of the following two
statements
\begin{align*}
    &``dd:=dd\ /\ 2; \\
    &\underline{if}\  dd \leq r\ \underline{do}\ r:= r - dd''
\end{align*}
operating on the variables ``$r$'' and ``$dd$'' leaves the relations
\begin{align} \label{eq:enumeration1}
    0 \leq r < dd
\end{align}
invariant. One just ``follows'' the little piece of program assuming that
(\ref{eq:enumeration1}) is satisfied to start with. After the execution of the
first statement, which halves the value of $dd$, but leaves $r$ unchanged,
the relations
\begin{align} \label{eq:enumeration2}
0 \leq r < 2*dd
\end{align}
will hold. Now we distinguish two mutually exclusive cases.

\noindent1) $dd \leq r$.\qquad Together with (\ref{eq:enumeration2}) this leads to the
relations
\begin{equation}
    dd \leq r < 2*dd\qquad;\label{eq:enumeration3}
\end{equation}
In this case the statement following \underline{do} will be executed,
ordering a decrease of $r$ by $dd$, so that from (\ref{eq:enumeration3})
it follows that eventually
\begin{equation*}
    0 \leq r < dd\qquad\qquad,
\end{equation*}
i.e. (\ref{eq:enumeration1}) will be satisfied.

\noindent2) \underline{non} $dd \leq r$ (i.e. $dd > r$).\qquad In this case the
statement following \underline{do} will be skipped and therefore also $r$ has
its final value. In this case ``$dd > r$'' together with \eqref{eq:enumeration2},
which is valid after the execution of the first statement leads immediately to
\begin{equation*}
    0 \leq r < dd
\end{equation*}
so that also in the second case \eqref{eq:enumeration1} will be satisfied.

Thus we have completed our proof of the in variance of relations
\eqref{eq:enumeration1}, we have also completed our example of enumerative
reasoning, conditional clauses included.