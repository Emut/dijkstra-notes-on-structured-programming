In the previous section we have stated that the programmer's duty is to
make his product ``usefully structured'' and we mentioned the program structure
in connection with a convincing demonstration of the correctness of the program.

But how do we convince? And how do we convince ourselves? What are the
typical patterns of thought enabling ourselves to understand? It is to a broad
survey of such questions that the current section is devoted. It is written with
my sincerest apologies to the professional psychologist, because it will be
amateurishly superficial. Yet I hope (and trust) that it will be sufficient to
give us a yardstick by which to measure the usefulness of a proposed structuring.

Among the mental aids available to understand a program (or a proof of its
correctness) there are three that I should like to mention explicitly:
\begin{enumerate}[1) ]
    \item Enumeration
    \item Mathematical induction
    \item Abstraction.
\end{enumerate}